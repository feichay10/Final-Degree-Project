En este capítulo se detalla los costes asociados al desarrollo del proyecto, incluyendo el tiempo estimado para cada tarea y el coste total del proyecto. Por lo que se refiere a gastos de hardware, software o licencias, no se han considerado ya que el proyecto se ha desarrollado utilizando herramientas de código abierto y simuladores que no requieren licencias. En cuanto al hardware utilizado, es el equipo personal que ya estaba disponible antes del inicio del proyecto.

\section{Costes del proyecto}
Para calcular los costes de personal, se ha considerado un salario anual de 28.000 euros brutos para un Ingeniero Informático Junior en España, con lo que el coste por hora sería de aproximadamente 15 euros brutos. Para el desarrollo del proyecto, se ha estimado un total de 300 horas de trabajo distribuidas en las siguientes tareas:
\begin{table}[H]
	\centering
	\small
	\begin{tabular}{|l|c|r|}
		\hline
		\textbf{Tarea}                   & \textbf{Horas estimadas} & \textbf{Coste estimado (euros)} \\
		\hline
		Investigación preliminar         & 20                       & 150 €                           \\ \hline
		Análisis de requisitos           & 20                       & 300 €                           \\ \hline
		Diseño de la red                 & 40                       & 600 €                           \\ \hline
		Selección de hardware y software & 20                       & 300 €                           \\ \hline
		Implementación en simuladores    & 100                      & 1500 €                          \\ \hline
		Pruebas y ajustes                & 50                       & 750 €                           \\ \hline
		Redacción de la memoria          & 50                       & 600 €                           \\ \hline
		\textbf{Total}                   & \textbf{300}             & \textbf{4200 €}                 \\
		\hline
	\end{tabular}
	\caption{Costes de personal}
	\label{tab:costes_proyecto}
\end{table}
El coste total del proyecto se muestra en la tabla \ref{tab:presupuesto}.
\begin{table}[H]
	\centering
	\small
	\begin{tabular}{|l|c|r|}
		\hline
		\textbf{Partida}                     & \textbf{Coste (euros)} \\
		\hline
		Licencias de software                & 0 €                    \\ \hline
		Amortización de equipos informáticos & 72 €                   \\ \hline
		Costes de personal                   & 4200 €                 \\ \hline
		Otros gastos                         & 50 €                   \\ \hline \hline
		Subtotal                             & 4322 €                 \\ \hline
		IGIC                                 & 302,54 €               \\ \hline \hline
		\textbf{Total}                       & \textbf{4624,54 €}     \\
		\hline
	\end{tabular}
	\caption{Presupuesto.}
	\label{tab:presupuesto}
\end{table}
