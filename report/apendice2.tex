\section{Configuración de los routers de la red de ISP}
Aquí se detalla la configuración de los routers utilizados en la simulación de la red de ISP, incluyendo los routers PE (Provider Edge), P (Provider) y CE (Customer Edge). 
\subsection{Configuración de los routers PE}
\label{Apendice2:configuracion_routers_pe}

\subsubsection*{Router PE1}
\begin{lstlisting}[language=RouterOS]
/system identity set name=PE1
/interface bridge add name=lo0
/ip address add address=1.1.1.11/32 interface=lo0 network=1.1.1.11
/ip address add address=10.0.0.1/30 interface=ether1 network=10.0.0.0
/ip address add address=172.16.0.1/30 interface=ether2 network=172.16.0.0
/ip address add address=10.0.0.13/30 interface=ether3 network=10.0.0.12

/routing ospf instance add name=backbone router-id=1.1.1.11
/routing ospf area add name=backbone area-id=0.0.0.0 inst=backbone
/routing ospf interface-template add interface=lo0 network=1.1.1.11/32 area=backbone
/routing ospf interface-template add interface=ether1 network=10.0.0.1/30 area=backbone

/mpls ldp add afi=ip lsr-id=1.1.1.11 transport-addresses=1.1.1.11
/mpls ldp interface add interface=lo0
/mpls ldp interface add interface=ether1
/mpls settings set dynamic-label-range=10000-11999

/routing bgp template set default address-families=ip,vpnv4 as=65000 router-id=1.1.1.11
/routing bgp connection
add connect=yes disabled=no listen=yes local.address=1.1.1.11 .role=ibgp \
    name=toPE2 remote.address=1.1.1.12 .as=65000 templates=default
/routing bgp connection
add connect=yes disabled=no listen=yes local.address=1.1.1.11 .role=ibgp \
    name=toPE3 remote.address=1.1.1.13 .as=65000 templates=default
/routing bgp connection
add connect=yes disabled=no listen=yes local.address=1.1.1.11 .role=ibgp \
    name=toPE4 remote.address=1.1.1.14 .as=65000 templates=default

/ip vrf add name=CE1 interfaces=ether2 

/routing bgp vpn
add export.redistribute=connected,static,bgp .route-targets=65000:100 \
    import.route-targets=65000:100 label-allocation-policy=per-vrf name=\
    bgp-mpls-vpn-1 route-distinguisher=65000:100 vrf=CE1

/routing bgp connection
add as=65000 connect=yes disabled=yes listen=yes local.address=172.16.0.1 \
    .role=ebgp name=toCE1 output.default-originate=always remote.address=\
    172.16.0.2 .as=65500 router-id=1.1.1.11 routing-table=CE1 vrf=CE1

/routing bgp connection add as=65000 connect=yes disabled=yes listen=yes local.address=172.16.0.1 .role=ebgp name=toCE1 output.default-originate=always remote.address=172.16.0.2 .as=65500 router-id=1.1.1.11 routing-table=CE1 vrf=CE1

/ip firewall mangle
add action=mark-routing chain=prerouting dst-address=!192.168.0.0/16 \
    in-interface=ether2 new-routing-mark=main passthrough=yes
/ip route add gateway=172.16.0.2@CE1 routing-table=CE1
/ip route add gateway=10.0.0.14
/ip route add dst-address=172.16.0.0/30 gateway=CE1@CE1 routing-table=main
\end{lstlisting}

\subsubsection*{Router PE2}
\begin{lstlisting}[language=RouterOS]
/system identity set name=PE2
/interface bridge add name=lo0
/ip address add address=1.1.1.12/32 interface=lo0
/ip address add address=172.16.0.5/30 interface=ether1
/ip address add address=10.0.0.5/30 interface=ether2

/routing ospf instance add name=backbone router-id=1.1.1.12
/routing ospf area add name=backbone area-id=0.0.0.0 inst=backbone
/routing ospf interface-template add interface=lo0 network=1.1.1.12/32 area=backbone
/routing ospf interface-template add interface=ether2 network=10.0.0.5/30 area=backbone

/mpls ldp add afi=ip lsr-id=1.1.1.12 transport-addresses=1.1.1.12
/mpls ldp interface add interface=ether2
/mpls settings set dynamic-label-range=12000-13999

/routing bgp template set default address-families=ip,vpnv4 as=65000 router-id=1.1.1.12
/routing bgp connection add name=toPE1 template=default local.address=1.1.1.12 local.role=ibgp remote.address=1.1.1.11 remote.as=65000 connect=yes listen=yes
/routing bgp connection add name=toPE3 template=default local.address=1.1.1.12 local.role=ibgp remote.address=1.1.1.13 remote.as=65000 connect=yes listen=yes
/routing bgp connection add name=toPE4 template=default local.address=1.1.1.12 local.role=ibgp remote.address=1.1.1.14 remote.as=65000 connect=yes listen=yes

/ip vrf add name=CE2 interfaces=ether1 

/routing bgp vpn add route-distinguisher=65000:100 import.route-targets=65000:100 vrf=CE2 label-allocation-policy=per-vrf export.route-targets=65000:100 .redistribute=connected,static,bgp

/routing bgp connection add name=toCE2 router-id=1.1.1.12 as=65000 local.address=172.16.0.5 .role=ebgp remote.address=172.16.0.6 .as=65500 routing-table=CE2 vrf=CE2 connect=yes listen=yes output.default-originate=always
\end{lstlisting}

\subsubsection*{Router PE3}
\begin{lstlisting}[language=RouterOS]
/system identity set name=PE3
/interface bridge add name=lo0
/ip address add address=1.1.1.13/32 interface=lo0
/ip address add address=10.0.0.9/30 interface=ether1
/ip address add address=172.16.0.9/30 interface=ether2

/routing ospf instance add name=backbone router-id=1.1.1.13
/routing ospf area add name=backbone area-id=0.0.0.0 inst=backbone
/routing ospf interface-template add interface=lo0 network=1.1.1.13/32 area=backbone
/routing ospf interface-template add interface=ether1 network=10.0.0.9/30 area=backbone

/mpls ldp add afi=ip lsr-id=1.1.1.13 transport-addresses=1.1.1.13
/mpls ldp interface add interface=ether1
/mpls settings set dynamic-label-range=14000-15999

/routing bgp template set default address-families=ip,vpnv4 as=65000 router-id=1.1.1.13
/routing bgp connection add name=toPE1 template=default local.address=1.1.1.13 local.role=ibgp remote.address=1.1.1.11 remote.as=65000 connect=yes listen=yes
/routing bgp connection add name=toPE2 template=default local.address=1.1.1.13 local.role=ibgp remote.address=1.1.1.12 remote.as=65000 connect=yes listen=yes
/routing bgp connection add name=toPE4 template=default local.address=1.1.1.13 local.role=ibgp remote.address=1.1.1.14 remote.as=65000 connect=yes listen=yes

/ip vrf add name=CE3 interfaces=ether2 

/routing bgp vpn add route-distinguisher=65000:100 import.route-targets=65000:100 vrf=CE3 label-allocation-policy=per-vrf export.route-targets=65000:100 .redistribute=connected,static,bgp

/routing bgp connection add name=toCE3 router-id=1.1.1.13 as=65000 local.address=172.16.0.9 .role=ebgp remote.address=172.16.0.10 .as=65500 routing-table=CE3 vrf=CE3 connect=yes listen=yes output.default-originate=always
\end{lstlisting}

\subsection{Configuración de los routers P}
\label{Apendice2:configuracion_routers_p}

\subsubsection*{Router P1}
\begin{lstlisting}[language=RouterOS]
/system identity set name=P1
/interface bridge add name=lo0
/ip address add address=1.1.1.1/32 interface=lo0
/ip address add address=10.0.0.18/30 interface=ether1
/ip address add address=10.0.0.26/30 interface=ether2
/ip address add address=10.0.0.6/30 interface=ether3
/ip address add address=10.0.0.10/30 interface=ether4

/routing ospf instance add name=backbone router-id=1.1.1.1
/routing ospf area add name=backbone area-id=0.0.0.0 inst=backbone
/routing ospf interface-template add interface=lo0 network=1.1.1.1/32 area=backbone
/routing ospf interface-template add interface=ether1 network=10.0.0.18/30 area=backbone
/routing ospf interface-template add interface=ether2 network=10.0.0.26/30 area=backbone
/routing ospf interface-template add interface=ether3 network=10.0.0.6/30 area=backbone
/routing ospf interface-template add interface=ether4 network=10.0.0.10/30 area=backbone

/mpls ldp add afi=ip lsr-id=1.1.1.1 transport-addresses=1.1.1.1
/mpls ldp interface add interface=lo0
/mpls ldp interface add interface=ether1
/mpls ldp interface add interface=ether2
/mpls ldp interface add interface=ether3
/mpls ldp interface add interface=ether4
/mpls settings set dynamic-label-range=20000-21999
\end{lstlisting}

\subsubsection*{Router P2}
\begin{lstlisting}[language=RouterOS]
/system identity set name=P3
/interface bridge add name=lo0
/ip address add address=1.1.1.3/32 interface=lo0
/ip address add address=10.0.0.21/30 interface=ether1
/ip address add address=10.0.0.17/30 interface=ether2
/ip address add address=10.0.0.2/30 interface=ether3
/ip address add address=10.0.0.14/30 interface=ether4

/routing ospf instance add name=backbone router-id=1.1.1.3
/routing ospf area add name=backbone area-id=0.0.0.0 inst=backbone
/routing ospf interface-template add interface=lo0 network=1.1.1.3/32 area=backbone
/routing ospf interface-template add interface=ether1 network=10.0.0.21/30 area=backbone
/routing ospf interface-template add interface=ether2 network=10.0.0.17/30 area=backbone
/routing ospf interface-template add interface=ether3 network=10.0.0.2/30 area=backbone
/routing ospf interface-template add interface=ether4 network=10.0.0.14/30 area=backbone

/mpls ldp add afi=ip lsr-id=1.1.1.3 transport-addresses=1.1.1.3
/mpls ldp interface add interface=lo0
/mpls ldp interface add interface=ether1
/mpls ldp interface add interface=ether2
/mpls ldp interface add interface=ether3
/mpls ldp interface add interface=ether4
/mpls settings set dynamic-label-range=24000-25999
\end{lstlisting}

\subsubsection*{Router P3}
\begin{lstlisting}[language=RouterOS]
/system identity set name=P3
/interface bridge add name=lo0
/ip address add address=1.1.1.3/32 interface=lo0
/ip address add address=10.0.0.21/30 interface=ether1
/ip address add address=10.0.0.17/30 interface=ether2
/ip address add address=10.0.0.2/30 interface=ether3
/ip address add address=10.0.0.14/30 interface=ether4

/routing ospf instance add name=backbone router-id=1.1.1.3
/routing ospf area add name=backbone area-id=0.0.0.0 inst=backbone
/routing ospf interface-template add interface=lo0 network=1.1.1.3/32 area=backbone
/routing ospf interface-template add interface=ether1 network=10.0.0.21/30 area=backbone
/routing ospf interface-template add interface=ether2 network=10.0.0.17/30 area=backbone
/routing ospf interface-template add interface=ether3 network=10.0.0.2/30 area=backbone
/routing ospf interface-template add interface=ether4 network=10.0.0.14/30 area=backbone

/mpls ldp add afi=ip lsr-id=1.1.1.3 transport-addresses=1.1.1.3
/mpls ldp interface add interface=lo0
/mpls ldp interface add interface=ether1
/mpls ldp interface add interface=ether2
/mpls ldp interface add interface=ether3
/mpls ldp interface add interface=ether4
/mpls settings set dynamic-label-range=24000-25999
\end{lstlisting}

\subsection{Configuración de los routers CE}
\label{Apendice2:configuracion_routers_ce}

\subsubsection*{Router CE1}
\begin{lstlisting}[language=RouterOS]
/system identity set name=CE1
/interface bridge add name=lo0
/ip address add address=1.1.1.21/32 interface=lo0 network=1.1.1.21
/ip address add address=192.168.1.1/24 interface=ether1 network=192.168.1.0
/ip address add address=172.16.0.2/30 interface=ether2 network= 172.16.0.0

/ip firewall address-list add address=192.168.1.0/24 list=BGP_OUT
/ip firewall nat add action=masquerade chain=srcnat dst-address=!192.168.0.0/16 out-interface=ether2

/ip route add gateway=172.16.0.1

/routing bgp connection
add as=65500 connect=yes listen=yes local.address=172.16.0.2 .role=ebgp name=\
    toPE1 output.network=BGP_OUT remote.address=172.16.0.1 .as=65000 router-id=\
    1.1.1.21
\end{lstlisting}

\subsubsection*{Router CE2}
\begin{lstlisting}[language=RouterOS]
/system identity set name=CE3
/interface bridge add name=lo0
/ip address add address=1.1.1.23/32 interface=lo0
/ip address add address=172.16.0.10/30 interface=ether1
/ip address add address=192.168.3.1/24 interface=ether2

/ip firewall address-list add address=192.168.3.0/24 list=BGP_OUT
/routing bgp connection add name=toPE3 as=65500 router-id=1.1.1.23 local.address=172.16.0.10 .role=ebgp remote.address=172.16.0.9 remote.as=65000 output.network=BGP_OUT connect=yes listen=yes
\end{lstlisting}

\subsubsection*{Router CE3}
\begin{lstlisting}[language=RouterOS]
/system identity set name=CE3
/interface bridge add name=lo0
/ip address add address=1.1.1.23/32 interface=lo0
/ip address add address=172.16.0.10/30 interface=ether1
/ip address add address=192.168.3.1/24 interface=ether2

/ip firewall address-list add address=192.168.3.0/24 list=BGP_OUT
/routing bgp connection add name=toPE3 as=65500 router-id=1.1.1.23 local.address=172.16.0.10 .role=ebgp remote.address=172.16.0.9 remote.as=65000 output.network=BGP_OUT connect=yes listen=yes
\end{lstlisting}

\section{Configuracion de la Oficina Central}
\subsection{Configuración router CE1}
\label{Apendice2:configuracion_ce1}
Se presenta a continuación la configuración del router CE1, que incluye la creación de VLANs, la asignación de direcciones IP y la configuración de DHCP Relay para reenviar solicitudes al servidor DHCP ubicado en la DMZ.
\begin{lstlisting}[language=RouterOS]
# ------------------------------------------------------------
# Identidad
/system identity set name=CE1

# ------------------------------------------------------------
# Habilitar todas las interfaces físicas
/interface ethernet enable [find]

# ------------------------------------------------------------
# Crear Bonding con LACP (802.3ad) (EtherChannel)
/interface bonding add name=bond1 mode=802.3ad slaves=ether1,ether2 transmit-hash-policy=layer-2-and-3

# ------------------------------------------------------------
# VLANs sobre el bonding
/interface vlan add name=vlan10-datos vlan-id=10 interface=bond1 mtu=1480
/interface vlan add name=vlan20-voz   vlan-id=20 interface=bond1 mtu=1480
/interface vlan add name=vlan30-dmz   vlan-id=30 interface=bond1 mtu=1480

# ------------------------------------------------------------
# Asignar direcciones IPv6 a cada VLAN
/ipv6 address add address=2001:db8:1234:0100::1/64 interface=vlan10-datos
/ipv6 address add address=2001:db8:1234:0101::1/64 interface=vlan20-voz
/ipv6 address add address=2001:db8:1234:0102::1/64 interface=vlan30-dmz

# ------------------------------------------------------------
# Configuración VRRP en las VLANs
/interface vrrp add name=vrrp-datos interface=vlan10-datos vrid=10 priority=150 version=3 v3-protocol=ipv6 mtu=1480
/interface vrrp add name=vrrp-voz   interface=vlan20-voz   vrid=20 priority=150 version=3 v3-protocol=ipv6 mtu=1480
/interface vrrp add name=vrrp-dmz   interface=vlan30-dmz   vrid=30 priority=150 version=3 v3-protocol=ipv6 mtu=1480

# Direcciones virtuales VRRP
/ipv6 address add address=2001:db8:1234:0100::2/64 interface=vrrp-datos
/ipv6 address add address=2001:db8:1234:0101::2/64 interface=vrrp-voz
/ipv6 address add address=2001:db8:1234:0102::2/64 interface=vrrp-dmz

# ------------------------------------------------------------
# Conexion directa con CE1_Backup (enlace de sincronización)
/interface ethernet enable ether3
/ipv6 address add address=2001:db8:1234:0103::1/64 interface=ether3

# ------------------------------------------------------------
# DHCPv6 relay en VLANs
/ipv6 dhcp-relay add name=relay-datos interface=vlan10-datos dhcp-server=2001:db8:1234:0102::132 link-address=2001:db8:1234:0100::1 disabled=no
/ipv6 dhcp-relay add name=relay-voz   interface=vlan20-voz   dhcp-server=2001:db8:1234:0102::132 link-address=2001:db8:1234:0101::1 disabled=no

# ------------------------------------------------------------
# Neighbor Discovery (RA, DNS)
/ipv6 nd add interface=vlan10-datos advertise-dns=yes managed-address-configuration=yes other-configuration=yes ra-lifetime=1800 ra-preference=low dns=2001:db8:1234:0102::133,2001:db8:1234:0102::134
/ipv6 nd add interface=vlan20-voz   advertise-dns=yes managed-address-configuration=yes other-configuration=yes ra-lifetime=1800 ra-preference=low dns=2001:db8:1234:0102::133,2001:db8:1234:0102::134
/ipv6 nd add interface=vlan30-dmz   advertise-dns=yes dns=2001:db8:1234:0102::133,2001:db8:1234:0102::134

# ------------------------------------------------------------
# Firewall básico IPv6
/ipv6 firewall filter add chain=forward action=accept protocol=udp dst-port=53 comment="Permitir tráfico DNS"
/ipv6 firewall filter add chain=forward action=accept protocol=tcp dst-port=53 comment="Permitir tráfico DNS"
/ipv6 firewall filter add chain=forward action=accept protocol=udp dst-port=547 comment="Permitir tráfico DHCPv6"
/ipv6 firewall filter add chain=forward action=accept comment="Permitir comunicación entre VLANs"
/ipv6 firewall filter add chain=forward action=accept protocol=icmpv6 comment="Permitir tráfico ICMPv6 entre VLANs y servidor DNS"
/ipv6 firewall filter add chain=forward action=accept src-address=2001:db8:1234:0100::/64 dst-address=2001:db8:1234:0102::/64 comment="Datos → DMZ"
/ipv6 firewall filter add chain=forward action=accept src-address=2001:db8:1234:0101::/64 dst-address=2001:db8:1234:0102::/64 comment="Voz → DMZ"
/ipv6 firewall filter add chain=forward action=accept src-address=2001:db8:1234:0102::/64 dst-address=2001:db8:1234:0100::/64 comment="DMZ → Datos"
/ipv6 firewall filter add chain=forward action=accept src-address=2001:db8:1234:0102::/64 dst-address=2001:db8:1234:0101::/64 comment="DMZ → Voz"

# ------------------------------------------------------------
# Activar fast-path (si aplica)
/ip settings set allow-fast-path=yes

# ------------------------------------------------------------
# Guardar configuración
/system backup save name=CE1
\end{lstlisting}

\subsection{Configuración del router CE1\_backup ARREGLAR}
\label{Apendice2:configuracion_ce1_backup}
\begin{lstlisting}[language=RouterOS]
# ------------------------------------------------------------
# Identidad
/system identity set name=CE1-Backup

# ------------------------------------------------------------
# Habilitar todas las interfaces físicas
/interface ethernet enable [find]

# ------------------------------------------------------------
# Crear Bonding con LACP (802.3ad) (EtherChannel)
/interface bonding add name=bond1 mode=802.3ad slaves=ether1,ether2 transmit-hash-policy=layer-2-and-3

# ------------------------------------------------------------
# VLANs sobre el bonding
/interface vlan add name=vlan10-datos vlan-id=10 interface=bond1 mtu=1480
/interface vlan add name=vlan20-voz   vlan-id=20 interface=bond1 mtu=1480
/interface vlan add name=vlan30-dmz   vlan-id=30 interface=bond1 mtu=1480

# ------------------------------------------------------------
# Asignar direcciones IPv6 a cada VLAN
/ipv6 address add address=2001:db8:1234:0100::1/64 interface=vlan10-datos
/ipv6 address add address=2001:db8:1234:0101::1/64 interface=vlan20-voz
/ipv6 address add address=2001:db8:1234:0102::1/64 interface=vlan30-dmz

# ------------------------------------------------------------
# Configuración VRRP en las VLANs
/interface vrrp add name=vrrp-datos interface=vlan10-datos vrid=10 priority=100 version=3 v3-protocol=ipv6 mtu=1480
/interface vrrp add name=vrrp-voz   interface=vlan20-voz   vrid=20 priority=100 version=3 v3-protocol=ipv6 mtu=1480 
/interface vrrp add name=vrrp-dmz   interface=vlan30-dmz   vrid=30 priority=100 version=3 v3-protocol=ipv6 mtu=1480

# Direcciones virtuales VRRP
/ipv6 address add address=2001:db8:1234:0100::2/64 interface=vrrp-datos
/ipv6 address add address=2001:db8:1234:0101::2/64 interface=vrrp-voz
/ipv6 address add address=2001:db8:1234:0102::2/64 interface=vrrp-dmz

# ------------------------------------------------------------
# Conexion directa con CE (enlace de sincronización)
/interface ethernet enable ether3
/ipv6 address add address=2001:db8:1234:0103::2/64 interface=ether3

# ------------------------------------------------------------
# DHCPv6 relay en VLANs
/ipv6 dhcp-relay add name=relay-datos interface=vlan10-datos dhcp-server=2001:db8:1234:0102::132 link-address=2001:db8:1234:0100::1 disabled=no
/ipv6 dhcp-relay add name=relay-voz   interface=vlan20-voz   dhcp-server=2001:db8:1234:0102::132 link-address=2001:db8:1234:0101::1 disabled=no

# ------------------------------------------------------------
# Neighbor Discovery (RA, DNS)
/ipv6 nd add interface=vlan10-datos advertise-dns=yes managed-address-configuration=yes other-configuration=yes ra-lifetime=1800 ra-preference=low dns=2001:db8:1234:0102::133,2001:db8:1234:0102::134
/ipv6 nd add interface=vlan20-voz   advertise-dns=yes managed-address-configuration=yes other-configuration=yes ra-lifetime=1800 ra-preference=low dns=2001:db8:1234:0102::133,2001:db8:1234:0102::134
/ipv6 nd add interface=vlan30-dmz   advertise-dns=yes dns=2001:db8:1234:0102::133,2001:db8:1234:0102::134

# ------------------------------------------------------------
# Firewall básico IPv6
/ipv6 firewall filter add chain=forward action=accept protocol=udp dst-port=53 comment="Permitir tráfico DNS"
/ipv6 firewall filter add chain=forward action=accept protocol=tcp dst-port=53 comment="Permitir tráfico DNS"
/ipv6 firewall filter add chain=forward action=accept protocol=udp dst-port=547 comment="Permitir tráfico DHCPv6"
/ipv6 firewall filter add chain=forward action=accept comment="Permitir comunicación entre VLANs"
/ipv6 firewall filter add chain=forward action=accept protocol=icmpv6 comment="Permitir tráfico ICMPv6 entre VLANs y servidor DNS"
/ipv6 firewall filter add chain=forward action=accept src-address=2001:db8:1234:0100::/64 dst-address=2001:db8:1234:0102::/64 comment="Datos → DMZ"
/ipv6 firewall filter add chain=forward action=accept src-address=2001:db8:1234:0101::/64 dst-address=2001:db8:1234:0102::/64 comment="Voz → DMZ"
/ipv6 firewall filter add chain=forward action=accept src-address=2001:db8:1234:0102::/64 dst-address=2001:db8:1234:0100::/64 comment="DMZ → Datos"
/ipv6 firewall filter add chain=forward action=accept src-address=2001:db8:1234:0102::/64 dst-address=2001:db8:1234:0101::/64 comment="DMZ → Voz"

# ------------------------------------------------------------
# Activar fast-path (si aplica)
/ip settings set allow-fast-path=yes

# ------------------------------------------------------------
# Guardar configuración
/system backup save name=CE1-Backup
\end{lstlisting}

\subsection{Docker Compose para los servicios de red}
\label{Apendice2:docker_compose}
\begin{lstlisting}[language=yaml]
  services:
  dns-primary-ipv6:
    build:
      context: ./dns-primary-ipv6
      dockerfile: Dockerfile
    image: dns-primary-ipv6:latest
    container_name: dns-primary-ipv6
    hostname: dns-primary-ipv6
    networks:
      services_net:
        ipv6_address: 2001:db8:1234:0102::133
    volumes:
      - ./dns-primary-ipv6/named.conf.options:/etc/bind/named.conf.options
      - ./dns-primary-ipv6/named.conf.local:/etc/bind/named.conf.local
      - ./dns-primary-ipv6/zones/:/etc/bind/zones/
    cap_add:
      - NET_ADMIN
    privileged: true
    restart: unless-stopped

  dns-secondary-ipv6:
    build:
      context: ./dns-secondary-ipv6
      dockerfile: Dockerfile
    image: dns-secondary-ipv6:latest
    container_name: dns-secondary-ipv6
    hostname: dns-secondary-ipv6
    networks:
      services_net:
        ipv6_address: 2001:db8:1234:0102::134
    volumes:
      - ./dns-secondary-ipv6/named.conf.options:/etc/bind/named.conf.options
      - ./dns-secondary-ipv6/named.conf.local:/etc/bind/named.conf.local
      - ./dns-secondary-ipv6/zones/:/etc/bind/zones/
    cap_add:
      - NET_ADMIN
    privileged: true
    restart: unless-stopped

  dhcp6-server:
    build:
      context: ./dhcp6-server
      dockerfile: Dockerfile
    image: dhcp6-server:latest
    container_name: dhcp6-server
    hostname: dhcp6-server
    networks:
      services_net:
        ipv6_address: 2001:db8:1234:0102::132
    volumes:
      - ./dhcp6-server/dhcpd6.conf:/etc/dhcp/dhcpd6.conf
      - ./dhcp6-server/isc-dhcp-server:/etc/default/isc-dhcp-server
    cap_add:
      - NET_ADMIN
    privileged: true
    restart: unless-stopped

networks:
  services_net:
    driver: bridge
    enable_ipv6: true
    ipam:
      driver: default
      config:
        - subnet: "2001:db8:1234:0102::/64"
\end{lstlisting}

\subsection{Servidor DHCP}
Aqui se presenta la configuración del servidor DHCP, tanto el Dockerfile como los archivos de configuración.

\newpage

\subsubsection{Dockerfile para el servidor DHCP}
\label{Apendice2:dockerfile_dhcp}
\begin{lstlisting}[language=Dockerfile]
FROM ubuntu:20.04

# Evitar prompts interactivos durante la instalación
ENV DEBIAN_FRONTEND=noninteractive

# Actualizar e instalar paquetes necesarios
RUN apt-get update && apt-get install -y \
    isc-dhcp-server \
    iputils-ping \
    net-tools \
    iproute2 \
    procps \
    && apt-get clean \
    && rm -rf /var/lib/apt/lists/*

# Crear directorio para logs y leases
RUN mkdir -p /var/log /var/lib/dhcp /var/run

# Crear archivo de leases inicial (requerido por dhcpd)
RUN touch /var/lib/dhcp/dhcpd6.leases

# Establecer permisos correctos
RUN chown -R dhcpd:dhcpd /var/lib/dhcp /var/log
RUN chmod 644 /var/lib/dhcp/dhcpd6.leases

# Copiar archivos de configuración del DHCP
COPY dhcpd6.conf /etc/dhcp/dhcpd6.conf
COPY isc-dhcp-server /etc/default/isc-dhcp-server

# Establecer permisos correctos para los archivos de configuración
RUN chmod 644 /etc/dhcp/dhcpd6.conf
RUN chmod 644 /etc/default/isc-dhcp-server

# Crear script de inicio
RUN echo '#!/bin/bash\n\
# Configurar IP estática dinámicamente\n\
ip -6 addr add 2001:db8:1234:0102::132/64 dev eth0\n\
ip -6 route add default via 2001:db8:1234:0102::1\n\
echo "nameserver 2001:db8:1234:0102::133" > /etc/resolv.conf\n\
echo "nameserver 2001:db8:1234:0102::134" >> /etc/resolv.conf\n\
# Iniciar servidor DHCP\n\
exec dhcpd -6 -cf /etc/dhcp/dhcpd6.conf -d eth0; exec bash' > /start.sh

RUN chmod +x /start.sh

# Crear usuario dhcpd si no existe
RUN useradd -r -s /bin/false dhcpd 2>/dev/null || true

# Exponer puerto DHCP
EXPOSE 547/udp

# Usar el script de inicio
CMD ["/start.sh"] 
\end{lstlisting}

\subsubsection{Archivo dhcpd6.conf}
\label{Apendice2:dhcpd6.conf}
\begin{lstlisting}[language=bash]
default-lease-time 600;
max-lease-time 7200;
authoritative;

option dhcp6.domain-search "cazg.es";
option dhcp6.name-servers 2001:db8:1234:0102::133, 2001:db8:1234:0102::134;

# VLAN Datos: 2001:db8:1234:0100::/64
subnet6 2001:db8:1234:0100::/64  {
    range6 2001:db8:1234:0100::2 2001:db8:1234:0100::ffff;
}

# VLAN Voz: 2001:db8:1234:0101::/64
subnet6 2001:db8:1234:0101::/64 {
    range6 2001:db8:1234:0101::2 2001:db8:1234:0101::ffff;
}

# VLAN DMZ: 2001:db8:1234:0102::/64
subnet6 2001:db8:1234:0102::/64 { }
\end{lstlisting}

\subsection{Servidor DNS Primario}
Aqui se presenta la configuración del servidor DNS primario, tanto el Dockerfile como los archivos de configuración.

\subsubsection{Dockerfile para el servidor DNS primario}
\label{Apendice2:dockerfile_dns_primario}
\begin{lstlisting}[language=Dockerfile]
FROM ubuntu:20.04

# Evitar prompts interactivos durante la instalación
ENV DEBIAN_FRONTEND=noninteractive

# Actualizar e instalar paquetes necesarios
RUN apt-get update && apt-get install -y \
    bind9 \
    bind9utils \
    bind9-doc \
    iputils-ping \
    net-tools \
    ifupdown \
    && apt-get clean \
    && rm -rf /var/lib/apt/lists/*

# Crear directorios necesarios
RUN mkdir -p /etc/bind/zones /var/cache/bind /var/lib/bind /var/log/named

# Copiar archivos de configuración DNS
COPY named.conf.options /etc/bind/
COPY named.conf.local /etc/bind/
COPY zones/ /etc/bind/zones/

# Establecer permisos correctos para bind
RUN chown -R bind:bind /etc/bind/zones /var/cache/bind /var/lib/bind /var/log/named
RUN chmod 755 /etc/bind/zones
RUN chmod 775 /var/cache/bind
RUN chmod 644 /etc/bind/named.conf.options /etc/bind/named.conf.local
RUN chmod 644 /etc/bind/zones/*

# Crear script de inicio
RUN echo '#!/bin/bash\n\
# Configurar IP estática dinámicamente\n\
ip addr add 192.168.1.133/28 dev eth0\n\
ip route add default via 192.168.1.131\n\
echo "nameserver 192.168.1.133" > /etc/resolv.conf\n\
echo "nameserver 192.168.1.134" >> /etc/resolv.conf\n\
# Iniciar servidor DNS\n\
exec named -g -c /etc/bind/named.conf -u bind; exec bash' > /start.sh

RUN chmod +x /start.sh

# Exponer puertos DNS
EXPOSE 53/udp 53/tcp

# Comando por defecto
CMD ["/start.sh"] 
\end{lstlisting}

\subsubsection{Archivo /etc/bind/named.conf.options}
\label{Apendice2:named.conf.options}
\begin{lstlisting}[language=bash]
options {
    directory "/var/cache/bind";

    // Solo escuchar en IPv6
    listen-on-v6 { 2001:db8:1234:0102::133; };
    listen-on { none; }; // Deshabilitar IPv4

    // Habilitar recursión
    recursion yes;

    // Permitir consultas solo desde red local
    allow-query {
        2001:db8:1234::/48;
        localhost;
    };

    // Permitir recursión solo a clientes autorizados
    allow-recursion {
        2001:db8:1234::/48;
        localhost;
    };

    // Permitir caché de DNS para red local
    allow-cache {
        2001:db8:1234::/48;
        localhost;
    };

    // Redirigir consultas externas a resolvers públicos (opcional pero recomendado)
    forwarders {
        2001:4860:4860::8888;
        2001:4860:4860::8844;
    };

    // Validación DNSSEC
    dnssec-validation auto;

    // No anunciarse como autoritativo en respuestas NXDOMAIN
    auth-nxdomain no;
};

\end{lstlisting}

\subsubsection{Archivo /etc/bind/named.conf.local}
\label{Apendice2:named.conf.local}
\begin{lstlisting}[language=bash]
// Zona directa principal - nombre -> IP
zone "cazg.es" {
    type master;
    file "/etc/bind/zones/db.cazg.es";
    allow-transfer { 2001:db8:1234:0102::134; };  // ns2 IP IPv6
    notify yes;
    also-notify { 2001:db8:1234:0102::134; };
};

// Zona inversa para red de servicios (2001:db8:1234:0102::/64)
zone "2.0.1.0.4.3.2.1.8.b.d.0.1.0.0.2.ip6.arpa" {
    type master;
    file "/etc/bind/zones/db.2001.db8.1234.0102"; 
    allow-transfer { 2001:db8:1234:0102::134; };  // ns2 IP IPv6
    notify yes;
    also-notify { 2001:db8:1234:0102::134; };
}; 
\end{lstlisting}

\subsubsection{Archivo /etc/bind/zones/db.cazg.es}
\label{Apendice2:db.cazg.es}
\begin{lstlisting}[language=bash]
$TTL    60
@       IN      SOA     ns1.cazg.es. admin.cazg.es. (
                        2025070102         ; Serial 
                        60                 ; Refresh
                        60                 ; Retry
                        60                 ; Expire
                        60 )               ; TTL mínimo

; Servidores de nombres
        IN      NS      ns1.cazg.es.
        IN      NS      ns2.cazg.es.

; Servidores DNS e infraestructura local
ns1     IN      AAAA    2001:db8:1234:0102::133
ns2     IN      AAAA    2001:db8:1234:0102::134
dhcp    IN      AAAA    2001:db8:1234:0102::132
\end{lstlisting}

\subsubsection{Archivo /etc/bind/zones/db.2001.db8.1234.0102}
\label{Apendice2:db.2001.db8.1234.0102}
\begin{lstlisting}[language=bash]
$TTL    60
@       IN      SOA     ns1.cazg.es. admin.cazg.es. (
                        2025070102      ; Serial
                        60              ; Refresh
                        60              ; Retry
                        60              ; Expire
                        60 )            ; TTL mínimo

; Servidores de nombres
        IN      NS      ns1.cazg.es.
        IN      NS      ns2.cazg.es.

; Registros PTR para resolución inversa - Red de servicios (2001:db8:1234:0102::/64)

; Infraestructura DNS y DHCP
2.3.1.0.0.0.0.0.0.0.0.0.0.0.0.0.2.0.1.0.4.3.2.1.8.b.d.0.1.0.0.2.ip6.arpa. IN PTR dhcp.cazg.es.
3.3.1.0.0.0.0.0.0.0.0.0.0.0.0.0.2.0.1.0.4.3.2.1.8.b.d.0.1.0.0.2.ip6.arpa. IN PTR ns1.cazg.es.
4.3.1.0.0.0.0.0.0.0.0.0.0.0.0.0.2.0.1.0.4.3.2.1.8.b.d.0.1.0.0.2.ip6.arpa. IN PTR ns2.cazg.es.
\end{lstlisting}

\subsection{Servidor DNS Secundario}
Aqui se presenta la configuración del servidor DNS secundario, tanto el Dockerfile como los archivos de configuración.

\subsubsection{Dockerfile para el servidor DNS secundario}
\label{Apendice2:dockerfile_dns_secundario}
\begin{lstlisting}[language=Dockerfile]
FROM ubuntu:20.04

# Evitar prompts interactivos durante la instalación
ENV DEBIAN_FRONTEND=noninteractive

# Actualizar e instalar paquetes necesarios
RUN apt-get update && apt-get install -y \
    bind9 \
    bind9utils \
    bind9-doc \
    iputils-ping \
    net-tools \
    ifupdown \
    && apt-get clean \
    && rm -rf /var/lib/apt/lists/*

# Crear directorios necesarios
RUN mkdir -p /etc/bind/zones /var/cache/bind /var/lib/bind /var/log/named

# Copiar archivos de configuración DNS
COPY named.conf.options /etc/bind/
COPY named.conf.local /etc/bind/

# Establecer permisos correctos para bind
RUN chown -R bind:bind /etc/bind/zones /var/cache/bind /var/lib/bind /var/log/named
RUN chmod 755 /etc/bind/zones
RUN chmod 775 /var/cache/bind
RUN chmod 644 /etc/bind/named.conf.options /etc/bind/named.conf.local

# Crear script de inicio
RUN echo '#!/bin/bash\n\
# Configurar IP estática dinámicamente\n\
ip addr add 192.168.1.134/28 dev eth0\n\
ip route add default via 192.168.1.131\n\
echo "nameserver 192.168.1.133" > /etc/resolv.conf\n\
echo "nameserver 192.168.1.134" >> /etc/resolv.conf\n\
# Iniciar servidor DNS\n\
exec named -g -c /etc/bind/named.conf -u bind; exec bash' > /start.sh

RUN chmod +x /start.sh

# Exponer puertos DNS
EXPOSE 53/udp 53/tcp

# Comando por defecto
CMD ["/start.sh"] 
\end{lstlisting}

\subsubsection{Archivo /etc/bind/named.conf.options}
\label{Apendice2:named.conf.options_dns2}
\begin{lstlisting}[language=bash]
options {
    directory "/var/cache/bind";

    // ns2 escucha solo en su propia IP
    listen-on-v6 { 2001:db8:1234:0102::134; };
    listen-on { none; };

    // Permitir consultas solo desde red local
    allow-query {
        2001:db8:1234::/48;
        localhost;
    };

    // Permitir recursión para red local
    recursion yes;
    allow-recursion {
        2001:db8:1234::/48;
        localhost;
    };

    // Permitir uso del caché DNS
    allow-cache {
        2001:db8:1234::/48;
        localhost;
    };

    // Redirigir consultas externas si no es autoritativo
    forwarders {
        2001:4860:4860::8888;
        2001:4860:4860::8844;
    };

    // Seguridad
    dnssec-validation auto;
    auth-nxdomain no;
};

\end{lstlisting}

\subsubsection{Archivo /etc/bind/named.conf.local}
\label{Apendice2:named.conf.local_dns2}
\begin{lstlisting}[language=bash]
// Zona directa principal - nombre -> IP
zone "cazg.es" {
    type slave;
    file "/etc/bind/zones/db.cazg.es";
    masters { 2001:db8:1234:0102::133; };  // ns1 IP IPv6
};

// Zona inversa para red de servicios (2001:db8:1234:0102::/64)
zone "2.0.1.0.4.3.2.1.8.b.d.0.1.0.0.2.ip6.arpa" {
    type slave;
    file "/etc/bind/zones/db.2001.db8.1234.0102"; 
    masters { 2001:db8:1234:0102::133; };  // ns1 IP IPv6
}; 
\end{lstlisting}

\subsection{Configuración del contenedor Docker para pruebas de red}
\label{Apendice2:configuracion_contenedor_red}
Aqui se presenta la configuración del contenedor Docker para pruebas de red, incluyendo el Dockerfile, el script de entrada y el archivo de configuración de Docker Compose.

\subsubsection{Dockerfile para el contenedor de pruebas de red}
\label{Apendice2:dockerfile_network_test}
\begin{lstlisting}[language=Dockerfile]
FROM debian:bookworm-slim

RUN apt-get update && apt-get install -y --no-install-recommends \
    iproute2 \
    iputils-ping \
    curl \
    tcpdump \
    net-tools \
    dnsutils \
    traceroute \
    iputils-tracepath \
    vim \
    bash \
    isc-dhcp-client \
    && apt-get clean \
    && rm -rf /var/lib/apt/lists/*

COPY entrypoint.sh /entrypoint.sh
RUN chmod +x /entrypoint.sh

ENTRYPOINT ["/entrypoint.sh"]
\end{lstlisting}

\subsubsection{Script de entrada para el contenedor de pruebas de red}
\label{Apendice2:entrypoint_network_test}
\begin{lstlisting}[language=bash]
#!/bin/bash

# Forzar resolución DNS al arranque
echo "nameserver 2001:db8:1234:0102::132" > /etc/resolv.conf
echo "nameserver 2001:db8:1234:0102::133" >> /etc/resolv.conf

# Quedarse activo
exec bash
\end{lstlisting}

\subsubsection{Docker Compose para el contenedor de pruebas de red}
\label{Apendice2:docker_compose_network_test}
\begin{lstlisting}[language=Dockerfile]
services:
pc-docker:
  container_name: pc-docker
  build:
    context: .
    dockerfile: Dockerfile
  privileged: true
  network_mode: bridge
  sysctls:
    net.ipv6.conf.all.disable_ipv6: 0
    net.ipv6.conf.default.disable_ipv6: 0
  restart: unless-stopped
\end{lstlisting}

\section{Configuración de sedes remotas e ISP}
\label{Apendice2:configuracion_sedes_remotas_isp}
\subsection{Configuración del router CE1 (Oficina Central)}
\label{Apendice2:configuracion_ce1_red_completa}
\begin{lstlisting}[language=RouterOS]
/interface bridge add name=lo0
/interface ethernet
set [ find default-name=ether1 ] disable-running-check=no
set [ find default-name=ether2 ] disable-running-check=no
set [ find default-name=ether3 ] disable-running-check=no
set [ find default-name=ether4 ] disable-running-check=no
set [ find default-name=ether5 ] disable-running-check=no
set [ find default-name=ether6 ] disable-running-check=no
set [ find default-name=ether7 ] disable-running-check=no
set [ find default-name=ether8 ] disable-running-check=no

/port set 0 name=serial0

/ip address add address=192.170.0.1 interface=lo0 network=192.170.0.1
/ip address add address=10.0.0.1/30 interface=ether4 network=10.0.0.0

# Conexion con el router CE1_backup
/interface ethernet enable ether3
/ip address add address=192.168.1.249/30 interface=ether3

# Crear Bonding con LACP (802.3ad) (EtherChannel)
/interface bonding add name=bond1 mode=802.3ad slaves=ether1,ether2 transmit-hash-policy=layer-2-and-3

# VLANs sobre el bonding
/interface vlan add name=vlan10-datos vlan-id=10 interface=bond1 mtu=1480
/interface vlan add name=vlan20-voz   vlan-id=20 interface=bond1 mtu=1480
/interface vlan add name=vlan30-dmz   vlan-id=30 interface=bond1 mtu=1480

# Asignar direcciones IP
/ip address add address=192.168.1.1/26 interface=vlan10-datos
/ip address add address=192.168.1.65/26 interface=vlan20-voz
/ip address add address=192.168.1.129/28 interface=vlan30-dmz

# Configuración VRRP en las VLANs
/interface vrrp add name=vrrp-datos interface=vlan10-datos vrid=10 priority=150
/interface vrrp add name=vrrp-voz interface=vlan20-voz vrid=20 priority=150
/interface vrrp add name=vrrp-dmz interface=vlan30-dmz vrid=30 priority=150

# Direcciones virtuales VRRP
/ip address add address=192.168.1.3/26 interface=vrrp-datos
/ip address add address=192.168.1.67/26 interface=vrrp-voz
/ip address add address=192.168.1.131/28 interface=vrrp-dmz

# DHCP relay
/ip dhcp-relay add name=relay-datos interface=vlan10-datos local-address=192.168.1.1 dhcp-server=192.168.1.132 disabled=no
/ip dhcp-relay add name=relay-voz interface=vlan20-voz local-address=192.168.1.65 dhcp-server=192.168.1.132 disabled=no
/ip dhcp-relay add name=relay-dmz interface=vlan30-dmz local-address=192.168.1.129 dhcp-server=192.168.1.132 disabled=no

# Firewall básico
/ip firewall filter add chain=forward action=accept protocol=udp src-port=67,68 dst-port=67,68 comment="Permitir tráfico DHCP"
/ip firewall filter add chain=forward action=accept comment="Permitir comunicación entre VLANs"

/ip dhcp-client add interface=ether1
# Exportar las redes de cliente a la VPN MPLS
/ip firewall address-list add address=192.168.1.0/26 list=BGP_OUT
/ip firewall address-list add address=192.168.1.64/26 list=BGP_OUT
/ip firewall address-list add address=192.168.1.128/28 list=BGP_OUT
/ip firewall nat add action=masquerade chain=srcnat dst-address=!192.168.0.0/16 out-interface=ether4

/ip route add gateway=10.0.0.2

/routing bgp connection
add as=65500 connect=yes listen=yes local.address=10.0.0.1 .role=ebgp name=\
    toPE2 output.network=BGP_OUT remote.address=10.0.0.2 .as=65000 router-id=\
    192.170.0.1

/system identity set name=CE1

# Asegurarte que ether1 está habilitado (por si acaso)
/interface ethernet set [find name=ether1] disabled=no

/system note set show-at-login=no
\end{lstlisting}

\subsection{Servidor DHCP}
\label{Apendice2:configuracion_dhcp_red_completa}
\subsubsection{Dockerfile para el servidor DHCP}
\begin{lstlisting}[language=RouterOS]
FROM ubuntu:20.04

# Evitar prompts interactivos durante la instalación
ENV DEBIAN_FRONTEND=noninteractive

# Actualizar e instalar paquetes necesarios
RUN apt-get update && apt-get install -y \
    isc-dhcp-server \
    iputils-ping \
    net-tools \
    iproute2 \
    procps \
    && apt-get clean \
    && rm -rf /var/lib/apt/lists/*

# Crear directorio para logs y leases
RUN mkdir -p /var/log /var/lib/dhcp /var/run

# Crear archivo de leases inicial (requerido por dhcpd)
RUN touch /var/lib/dhcp/dhcpd.leases

# Establecer permisos correctos
RUN chown -R dhcpd:dhcpd /var/lib/dhcp /var/log
RUN chmod 644 /var/lib/dhcp/dhcpd.leases

# Copiar archivos de configuración del DHCP
COPY dhcpd.conf /etc/dhcp/dhcpd.conf
COPY isc-dhcp-server /etc/default/isc-dhcp-server

# Establecer permisos correctos para los archivos de configuración
RUN chmod 644 /etc/dhcp/dhcpd.conf
RUN chmod 644 /etc/default/isc-dhcp-server

# Crear script de inicio
RUN echo '#!/bin/bash\n\
# Configurar IP estática dinámicamente\n\
ip addr add 192.168.1.132/28 dev eth0\n\
ip route add default via 192.168.1.131\n\
echo "nameserver 192.168.1.133" > /etc/resolv.conf\n\
echo "nameserver 192.168.1.134" >> /etc/resolv.conf\n\
# Iniciar servidor DHCP\n\
exec dhcpd -4 -cf /etc/dhcp/dhcpd.conf -d eth0; exec bash' > /start.sh

RUN chmod +x /start.sh

# Crear usuario dhcpd si no existe
RUN useradd -r -s /bin/false dhcpd 2>/dev/null || true

# Exponer puerto DHCP
EXPOSE 67/udp

# Usar el script de inicio
CMD ["/start.sh"] 
\end{lstlisting}

\subsubsection{Archivo dhcpd.conf}
\label{Apendice2:dhcpd.conf}
\begin{lstlisting}[language=bash]
default-lease-time 600;
max-lease-time 7200;
authoritative;

option domain-name "cazg.es"; # Dominio de la red
option domain-name-servers 192.168.1.133, 192.168.1.134; # IPs de los DNS primario y secundario

# Subred para VLAN 10 (Datos)
subnet 192.168.1.0 netmask 255.255.255.192 {
    range 192.168.1.4 192.168.1.62;
    option routers 192.168.1.3; 
    option broadcast-address 192.168.1.63;
}

# Subred para VLAN 20 (Voz)
subnet 192.168.1.64 netmask 255.255.255.192 {
    range 192.168.1.68 192.168.1.126;
    option routers 192.168.1.67;
    option broadcast-address 192.168.1.127;
}

# Subred para VLAN 30 (DMZ)
subnet 192.168.1.128 netmask 255.255.255.240 {
    # No se asignan IPs dinámicamente ya que son servidores con IPs fijas
    option routers 192.168.1.131; 
    option broadcast-address 192.168.1.143;
} 
\end{lstlisting}

\subsection{Servidores DNS}
\label{Apendice2:configuracion_dns_red_completa}
\subsubsection{Configuración del servidor DNS primario}
\label{Apendice2:configuracion_dns_primario}
\subsubsection*{Dockerfile para el servidor DNS primario}
\begin{lstlisting}[language=Dockerfile]
  FROM ubuntu:20.04

# Evitar prompts interactivos durante la instalación
ENV DEBIAN_FRONTEND=noninteractive

# Actualizar e instalar paquetes necesarios
RUN apt-get update && apt-get install -y \
    bind9 \
    bind9utils \
    bind9-doc \
    iputils-ping \
    net-tools \
    ifupdown \
    && apt-get clean \
    && rm -rf /var/lib/apt/lists/*

# Crear directorios necesarios
RUN mkdir -p /etc/bind/zones /var/cache/bind /var/lib/bind /var/log/named

# Copiar archivos de configuración DNS
COPY named.conf.options /etc/bind/
COPY named.conf.local /etc/bind/
COPY zones/ /etc/bind/zones/

# Establecer permisos correctos para bind
RUN chown -R bind:bind /etc/bind/zones /var/cache/bind /var/lib/bind /var/log/named
RUN chmod 755 /etc/bind/zones
RUN chmod 775 /var/cache/bind
RUN chmod 644 /etc/bind/named.conf.options /etc/bind/named.conf.local
RUN chmod 644 /etc/bind/zones/*

# Crear script de inicio
RUN echo '#!/bin/bash\n\
# Configurar IP estática dinámicamente\n\
ip addr add 192.168.1.133/28 dev eth0\n\
ip route add default via 192.168.1.131\n\
echo "nameserver 192.168.1.133" > /etc/resolv.conf\n\
echo "nameserver 192.168.1.134" >> /etc/resolv.conf\n\
# Iniciar servidor DNS\n\
exec named -g -c /etc/bind/named.conf -u bind; exec bash' > /start.sh

RUN chmod +x /start.sh

# Exponer puertos DNS
EXPOSE 53/udp 53/tcp

# Comando por defecto
CMD ["/start.sh"] 
\end{lstlisting}

\subsubsection*{Archivo /etc/bind/named.conf.options}
\label{Apendice2:named.conf.options_dns_primario}
\begin{lstlisting}[language=bash]
options {
    directory "/var/cache/bind";

    // Escucha solo en su IP privada
    listen-on { 192.168.1.133; };
    listen-on-v6 { none; };  // si no estás usando IPv6 aquí

    // Consultas permitidas desde red local
    allow-query {
        192.168.1.0/28;
        localhost;
    };

    // Reenvío a DNS externos para dominios no autoritativos
    forwarders {
        8.8.8.8;
        1.1.1.1;
    };

    // Habilitar recursión para la red local
    recursion yes;
    allow-recursion {
        192.168.1.0/28;
        localhost;
    };

    // Permitir uso del caché
    allow-cache {
        192.168.1.0/28;
        localhost;
    };

    dnssec-validation auto;
    auth-nxdomain no;
};
\end{lstlisting}

\subsubsection*{Archivo /etc/bind/named.conf.local}
\label{Apendice2:named.conf.local_dns_primario}
\begin{lstlisting}[language=bash]
// Zona directa nombre -> IP
zone "cazg.es" {
    type master;
    file "/etc/bind/zones/db.cazg.es";
    allow-transfer { 192.168.1.134; };  // ns2 IP privada 
};

// Zona inversa IP -> nombre
zone "1.168.192.in-addr.arpa" {
    type master;
    file "/etc/bind/zones/db.192.168.1"; // subred 192.168.1.0/28
    allow-transfer { 192.168.1.134; };  // ns2 IP privada
}; 
\end{lstlisting}

\subsubsection*{Archivo /etc/bind/zones/db.cazg.es}
\label{Apendice2:db.cazg.es_dns_primario}
\begin{lstlisting}[language=bash]
$TTL    60
@       IN      SOA     ns1.cazg.es. admin.cazg.es. (
                        2025051001         ; Serial
                        60                 ; Refresh
                        60                 ; Retry
                        60                 ; Expire
                        60 )               ; TTL

        IN      NS      ns1.cazg.es.
        IN      NS      ns2.cazg.es.
ns1     IN      A       192.168.1.133
ns2     IN      A       192.168.1.134
dhcp    IN      A       192.168.1.132 
\end{lstlisting}

\subsubsection*{Archivo /etc/bind/zones/db.192.168.1}
\label{Apendice2:db.192.168.1_dns_primario}
\begin{lstlisting}[language=bash]
$TTL    60
@       IN      SOA     ns1.cazg.es. admin.cazg.es. (
                        2025051001      ; Serial
                        60              ; Refresh
                        60              ; Retry
                        60              ; Expire
                        60 )            ; TTL

        IN      NS      ns1.cazg.es.
        IN      NS      ns2.cazg.es.
133     IN      PTR     ns1.cazg.es.
134     IN      PTR     ns2.cazg.es.
132     IN      PTR     dhcp.cazg.es.
\end{lstlisting}

\subsubsection{Configuración del servidor DNS secundario}
\label{Apendice2:configuracion_dns_secundario}
\subsubsection*{Dockerfile para el servidor DNS secundario}
\begin{lstlisting}[language=Dockerfile]
FROM ubuntu:20.04

# Evitar prompts interactivos durante la instalación
ENV DEBIAN_FRONTEND=noninteractive

# Actualizar e instalar paquetes necesarios
RUN apt-get update && apt-get install -y \
    bind9 \
    bind9utils \
    bind9-doc \
    iputils-ping \
    net-tools \
    ifupdown \
    && apt-get clean \
    && rm -rf /var/lib/apt/lists/*

# Crear directorios necesarios
RUN mkdir -p /etc/bind/zones /var/cache/bind /var/lib/bind /var/log/named

# Copiar archivos de configuración DNS
COPY named.conf.options /etc/bind/
COPY named.conf.local /etc/bind/

# Establecer permisos correctos para bind
RUN chown -R bind:bind /etc/bind/zones /var/cache/bind /var/lib/bind /var/log/named
RUN chmod 755 /etc/bind/zones
RUN chmod 775 /var/cache/bind
RUN chmod 644 /etc/bind/named.conf.options /etc/bind/named.conf.local

# Crear script de inicio
RUN echo '#!/bin/bash\n\
# Configurar IP estática dinámicamente\n\
ip addr add 192.168.1.134/28 dev eth0\n\
ip route add default via 192.168.1.131\n\
echo "nameserver 192.168.1.133" > /etc/resolv.conf\n\
echo "nameserver 192.168.1.134" >> /etc/resolv.conf\n\
# Iniciar servidor DNS\n\
exec named -g -c /etc/bind/named.conf -u bind; exec bash' > /start.sh

RUN chmod +x /start.sh

# Exponer puertos DNS
EXPOSE 53/udp 53/tcp

# Comando por defecto
CMD ["/start.sh"] 
\end{lstlisting}

\subsubsection*{Archivo /etc/bind/named.conf.options}
\begin{lstlisting}[language=bash]
options {
    directory "/var/cache/bind";

    // Escuchar en su propia IP
    listen-on { 192.168.1.134; };
    listen-on-v6 { none; };

    // Permitir consultas solo desde la red local
    allow-query {
        192.168.1.0/28;
        localhost;
    };

    // Permitir recursión desde la red local
    recursion yes;
    allow-recursion {
        192.168.1.0/28;
        localhost;
    };

    // Permitir uso de caché
    allow-cache {
        192.168.1.0/28;
        localhost;
    };

    // Reenvío a resolvers públicos
    forwarders {
        8.8.8.8;
        1.1.1.1;
    };

    dnssec-validation auto;
    auth-nxdomain no;
};
\end{lstlisting}

\subsubsection*{Archivo /etc/bind/named.conf.local}
\begin{lstlisting}[language=bash]
// Zona directa nombre -> IP
zone "cazg.es" {
    type slave;
    file "/etc/bind/zones/db.cazg.es";
    masters { 192.168.1.133; };  // ns1 IP privada
};

// Zona inversa IP -> nombre
zone "1.168.192.in-addr.arpa" {
    type slave;
    file "/etc/bind/zones/db.192.168.1"; // subred 192.168.1.0/28
    masters { 192.168.1.133; };  // ns1 IP privada
}; 
\end{lstlisting}

\subsection{Configuración del router CE2 (San Cristóbal)}
\label{Apendice2:configuracion_ce2_san_cristobal}
\begin{lstlisting}[language=RouterOS]
# Identidad
/system identity set name=CE2

# Habilitar interfaces físicas
/interface ethernet enable [find]

# VLANs sobre ether1
/interface vlan add interface=ether1 vlan-id=10 name=vlan10-datos
/interface vlan add interface=ether1 vlan-id=20 name=vlan20-voz

# Asignar direcciones IP
/ip address add address=192.168.2.1/26 interface=vlan10-datos
/ip address add address=192.168.2.33/26 interface=vlan20-voz

# Pool de direcciones DHCP
/ip pool add name=pool_datos ranges=192.168.2.2-192.168.2.30
/ip pool add name=pool_voz ranges=192.168.2.34-192.168.2.62

# Servidor DHCP para VLAN 10 (Datos)
/ip dhcp-server add name=dhcp_datos interface=vlan10-datos address-pool=pool_datos disabled=no
/ip dhcp-server network add address=192.168.2.0/26 gateway=192.168.2.1

# Servidor DHCP para VLAN 20 (Voz)
/ip dhcp-server add name=dhcp_voz interface=vlan20-voz address-pool=pool_voz disabled=no
/ip dhcp-server network add address=192.168.2.32/26 gateway=192.168.2.33

# Firewall: bloquear tráfico entre VLANs
/ip firewall filter add chain=forward action=drop src-address=192.168.2.0/26 dst-address=192.168.2.32/26 comment="Bloquear Datos -> Voz"
/ip firewall filter add chain=forward action=drop src-address=192.168.2.32/26 dst-address=192.168.2.0/26 comment="Bloquear Voz -> Datos"

# Habilitar IP forwarding
/ip settings set allow-fast-path=yes

# Direcciones IP para BGP e interconexión MPLS
/ip address add address=20.0.0.1/30 interface=ether2 network=20.0.0.0

# Loopback
/interface bridge add name=lo0
/ip address add address=192.170.0.5 interface=lo0 network=192.170.0.5

# Exportar las redes a la VPN MPLS
/ip firewall address-list add address=192.168.2.0/26 list=BGP_OUT
/ip firewall address-list add address=192.168.2.32/26 list=BGP_OUT

# Configuración eBGP con PE2
/routing bgp connection add name=toPE2 as=65500 router-id=192.170.0.5 \
    local.address=20.0.0.1 .role=ebgp remote.address=20.0.0.2 remote.as=65000 \
    output.network=BGP_OUT connect=yes listen=yes

# Asegurarte que ether1 está habilitada
/interface ethernet set [find name=ether1] disabled=no

# Guardar configuración final
/system backup save name=CE2
\end{lstlisting}

\section{Configuración de los switches}
Aqui se puede ver la configuración de los switches usados en las simulaciones de este proyecto.

\subsection{Configuración del switch de distribución}
\label{Apendice2:configuracion_switch_distribucion_central}
\begin{lstlisting}[language=CiscoIOS]
enable
configure terminal
hostname SW_Distribucion
no ip routing
spanning-tree mode rapid-pvst

! VLANs
vlan 10
  name Datos
exit
vlan 20
  name Voz
exit
vlan 30
  name DMZ
exit

! --------------------------
! EtherChannel 1: CE1 (Router Principal)
! Puertos: Gi0/0 y Gi0/1
interface range GigabitEthernet0/0 - 1
  description EtherChannel to Router CE1
  switchport trunk encapsulation dot1q
  switchport mode trunk
  channel-group 1 mode active
  spanning-tree portfast trunk
  no shutdown
exit

interface Port-channel1
  description Port-Channel to Router CE1
  switchport trunk encapsulation dot1q
  switchport mode trunk
  spanning-tree portfast trunk
  no shutdown
exit

! --------------------------
! EtherChannel 2: CE1_Backup (Router Secundario)
! Puertos: Gi0/2 y Gi0/3
interface range GigabitEthernet0/2 - 3
  description EtherChannel to Router CE1_Backup
  switchport trunk encapsulation dot1q
  switchport mode trunk
  channel-group 2 mode active
  spanning-tree portfast trunk
  no shutdown
exit

interface Port-channel2
  description Port-Channel to Router CE1_Backup
  switchport trunk encapsulation dot1q
  switchport mode trunk
  spanning-tree portfast trunk
  no shutdown
exit

! --------------------------
! EtherChannel 3: SW_Acceso_DMZ
! Puertos: Gi1/0 y Gi1/1
interface range GigabitEthernet1/0 - 1
  description EtherChannel to SW_Acceso_DMZ
  switchport trunk encapsulation dot1q
  switchport mode trunk
  channel-group 3 mode active
  no shutdown
exit

interface Port-channel3
  description Port-Channel to SW_Acceso_DMZ
  switchport trunk encapsulation dot1q
  switchport mode trunk
  no shutdown
exit

! --------------------------
! EtherChannel 4: SW_Acceso_LAN
! Puertos: Gi1/2 y Gi1/3
interface range GigabitEthernet1/2 - 3
  description EtherChannel to SW_Acceso_LAN
  switchport trunk encapsulation dot1q
  switchport mode trunk
  channel-group 4 mode active
  no shutdown
exit

interface Port-channel4
  description Port-Channel to SW_Acceso_LAN
  switchport trunk encapsulation dot1q
  switchport mode trunk
  no shutdown
exit

end
write memory
\end{lstlisting}

\subsection{Configuración del switch de acceso LAN}
\label{Apendice2:configuracion_switch_acceso_lan_central}
\begin{lstlisting}[language=CiscoIOS]
enable
configure terminal
hostname SW_Acceso_LAN
no ip routing
spanning-tree mode rapid-pvst

vlan 10
  name Datos
exit
vlan 20
  name Voz
exit

! Puertos de acceso VLAN 10
interface range GigabitEthernet0/2-3, GigabitEthernet1/1-3, GigabitEthernet2/0
  description VLAN Datos
  switchport mode access
  switchport access vlan 10
  no shutdown
exit

! Puertos de acceso VLAN 20
interface range GigabitEthernet2/1-3, GigabitEthernet3/0-3
  description VLAN Voz
  switchport mode access
  switchport access vlan 20
  no shutdown
exit

! ---- EtherChannel hacia SW_Distribucion ----
interface range GigabitEthernet0/0, GigabitEthernet0/1
  description Trunks hacia SW_Distribucion
  switchport trunk encapsulation dot1q
  switchport mode trunk
  channel-group 4 mode active
  no shutdown
exit

interface Port-channel4
  description EtherChannel a SW_Distribucion
  switchport trunk encapsulation dot1q
  switchport mode trunk
  no shutdown
exit

end
write memory
\end{lstlisting}

\subsection{Configuración del switch de acceso DMZ}
\label{Apendice2:configuracion_switch_acceso_dmz_central}
\begin{lstlisting}[language=CiscoIOS]
enable
configure terminal
hostname SW_Acceso_DMZ
no ip routing
spanning-tree mode rapid-pvst

vlan 30
  name DMZ
exit

! Puertos de acceso VLAN 30
interface range GigabitEthernet0/2-3, GigabitEthernet1/0-3
  description VLAN DMZ
  switchport mode access
  switchport access vlan 30
  no shutdown
exit

! ---- EtherChannel hacia SW_Distribucion ----
interface range GigabitEthernet0/0, GigabitEthernet0/1
  description Trunks hacia SW_Distribucion
  switchport trunk encapsulation dot1q
  switchport mode trunk
  channel-group 3 mode active
  no shutdown
exit

interface Port-channel3
  description EtherChannel a SW_Distribucion
  switchport trunk encapsulation dot1q
  switchport mode trunk
  no shutdown
exit

end
write memory
\end{lstlisting}

\section{Docker Compose para la FreePBX}
\label{Apendice2:docker_compose_freepbx}
\begin{lstlisting}[language=Dockerfile]
version: '3.8'

services:
  freepbx:
    image: tiredofit/freepbx:latest
    container_name: freepbx
    restart: always
    network_mode: "host"
    ports:
      - "80:80"         # Web interface HTTP
      - "443:443"       # Web interface HTTPS
      - "5060:5060/udp" # SIP UDP
      - "5160:5160/udp" # SIP UDP Alternative
      - "18000-18100:18000-18100/udp" # RTP Ports
    environment:
      - VIRTUAL_HOST=freepbx.local
      - RTP_START=18000
      - RTP_FINISH=18100
      - ASTERISKVERSION=18
      - DB_EMBEDDED=TRUE  # Usa base de datos SQLite embebida
    volumes:
      - ./data:/data
      - ./logs:/var/log
\end{lstlisting}